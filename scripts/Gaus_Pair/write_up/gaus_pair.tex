\documentclass[preprint,showpacs,preprintnumbers,amsmath,amssymb]{revtex4}
\usepackage{graphicx}% Include figure files
\usepackage{dcolumn}% Align table columns on decimal point
\usepackage{bm}% bold math
\usepackage{amsmath}
\usepackage{braket}
\def\fct#1{\mathop{\rm #1}} % e.g.,  \fct{tr}
\def\Rz{\mathbb{R}}
\def\Cz{\mathbb{C}}
\newcommand{\be}{\begin{equation}}
\newcommand{\ee}{\end{equation}}
\newcommand{\bea}{\begin{eqnarray}}
\newcommand{\eea}{\end{eqnarray}}
\newcommand{\pd}{\partial}
\newcommand{\bbx}{\bar{\mbox{\bf x}}^{ij}}
\newcommand{\bbG}{\left(\mbox{\bf G}^i+\mbox{\bf G}^j\right)}
\newcommand{\bx}{\mbox{\bf x}}        % centered radii
\newcommand{\by}{\mbox{\bf y}}
\newcommand{\bU}{\mbox{\bf U}}
\newcommand{\bV}{\mbox{\bf V}}
\newcommand{\bP}{\mbox{\bf P}}
\newcommand{\bW}{\mbox{\bf W}}
\newcommand{\uu}{\mathcal{U}}
\newcommand{\Eq}{Eq. \ref}
\newcommand{\Fig}{Fig. \ref}
\newcommand{\bR}{\mbox{\bf R}}
\newcommand{\bQ}{\mbox{\bf Q}}
\newcommand{\br}{\mbox{\bf r}}
\newcommand{\bz}{\mbox{\bf z}}
\newcommand{\bh}{\mbox{\bf h}}
\newcommand{\bq}{\mbox{\bf q}}
\newcommand{\bp}{\mbox{\bf p}}
\newcommand{\ba}{\mbox{\bf a}}
\newcommand{\bb}{\mbox{\bf b}}
\newcommand{\bc}{\mbox{\bf c}}
\newcommand{\bd}{\mbox{\bf d}}
\newcommand{\bl}{\mbox{\bf l}}
\newcommand{\bs}{\mbox{\bf s}}
\newcommand{\bu}{\mbox{\bf u}}
\newcommand{\bC}{\mbox{\bf C}}
\newcommand{\bS}{\mbox{\bf S}}
\newcommand{\bF}{\mbox{\bf F}}
\newcommand{\bA}{\mbox{\bf A}}
\newcommand{\bB}{\mbox{\bf B}}
\newcommand{\bD}{\mbox{\bf D}}
\newcommand{\bG}{\mbox{\bf G}}
\newcommand{\bM}{\mbox{\bf M}}
\newcommand{\bO}{\boldsymbol\Omega}
\newcommand{\bL}{\boldsymbol\Lambda}
\newcommand{\bZ}{\mbox{\bf Z}}
\newcommand{\bT}{\mbox{\bf T}}
\newcommand{\bK}{\mbox{\bf K}}
\newcommand{\bbf}{\mbox{\bf f}}
\newcommand{\bI}{\mbox{\bf I}}
\newcommand{\bg}{\mbox{\bf g}}
\def\eps{\varepsilon}
\def\<{\left\langle}                 % expectation
\def\>{\right\rangle}                 % expectation
\def\D{\displaystyle}               % display style
\def\half{{1\over 2}}               % display style
\def\att{                   % mark at the margin
    \marginpar[ \hspace*{\fill} \raisebox{-0.2em}{\rule{2mm}{1.2em}} ]
    {\raisebox{-0.2em}{\rule{2mm}{1.2em}} }
        }
\begin{document}

%===============================================================================%
\section*{Gaussian Pairwise Interations}
%===============================================================================%
Previously we were using a Lennard Jones pair-wise interaction to minimize our particle distributions. 
This pair-wise potential needs to be defined locally, if this interaction extends to long distances it will change the global distribution of the particles. 
Previously we used the Lennard-Jones Potential, however, in high-dimensions this will no longer be local, but become long-ranged. 

A better approach will be to use a gaussian potential for minimizing the pairwise interactions. 
Consider a quasi-Gaussian pairwise interaction ($\bU_{ij}$)
\be
\begin{split}
    \bU_{ij}(\bx_i,\bx_j)&:=-\exp\left\{\frac{\left(|\bx_{ij}|-\sigma_i\right)^2}{\left(\gamma\;\sigma_i\right)^2}\right\}\\
    |\bx_{ij}|&=\sum\left(\bx_i-\bx_j\right)^2
\end{split}
\ee
Where $\gamma$ is a constant defining the width of the Gaussian (can be paramaterized, try $\gamma$=0.1 for testing purposes). 



$\sigma$ represents the distance between nearest neighbors for our regularly distributed $\bP$ gridpoints. 
\be
\sigma_i=\sigma(\bx_i) := c\cdot \left[N\cdot P\left(\bx_i\right)\right]^{-1/d}
\ee

The constant c should be on the order of 1, ensuring particles do not form clusters (c is too small) or expand to the surface (c is too large). 
%===============================================================================%
\subsection*{Distribution of Interest}
%===============================================================================%
The distribution of interest ($\bP$) is defined in terms of the Morse Potential
\be
\begin{split}
    \bP(\bx)&:=\frac{E_{cut}-\bV(\bx)}{\int d\bx\; E_{cut}-\bV(\bx)}\\
    \bV(\bx)&:=D\sum_i \left(e^{-w_i\bx_i}-1\right)^2
\end{split}
\ee

Where we define a maximum energy contour ($E_{cut}$)
\[ \begin{cases} 
    \bP(\bx)=0 & \bV(\bx) > E_{cut} \\
    \bP(\bx)>0 & \bV(\bx) < E_{cut} \\
   \end{cases}
\]

$\sigma$ represents the distance between nearest neighbors for our regularly distributed $\bP$ gridpoints. 
\be
\sigma_i=\sigma(\bx_i) := c\cdot \left[N\cdot P\left(\bx_i\right)\right]^{-1/d}
\ee
\end{document}
